\documentclass[11pt]{article}
\usepackage{fullpage}
\usepackage{latexsym}
\usepackage{verbatim}
\usepackage{code,proof,amsthm,amssymb,amsmath,stmaryrd,tipa}
\usepackage{ifthen}
\usepackage{graphics}
\usepackage{mathpartir}
\usepackage{hyperref}
\usepackage{times}
\usepackage{url}

\newcommand\sml[1]{\texttt{#1}}

%% Question macros
\newcounter{question}[section]
\newcounter{extracredit}[section]
\newcounter{totalPoints}
\setcounter{totalPoints}{0}
\newcommand{\question}[1]
{
\bigskip
\addtocounter{question}{1}
\addtocounter{totalPoints}{#1}
\noindent
{\textbf{Task \thesection.\thequestion}}[#1 points]:


}
\newcommand{\ecquestion}
{
\bigskip
\addtocounter{extracredit}{1}
\noindent
\textbf{Extra Credit \thesection.\theextracredit}:}

%% Set to "false" to generate the problem set, set to "true" to generate the solution set
\def\issolution{false}

\newcounter{taskcounter}
\newcounter{taskPercentCounter}
\newcounter{taskcounterSection}
\setcounter{taskcounter}{1}
\setcounter{taskPercentCounter}{0}
\setcounter{taskcounterSection}{\value{section}}
\newcommand{\mayresettaskcounter}{\ifthenelse{\value{taskcounterSection} < \value{section}}
{\setcounter{taskcounterSection}{\value{section}}\setcounter{taskcounter}{1}}{}}

% Solution-only - uses an "input" so that it's still safe to publish the problem set file
\definecolor{solutioncolor}{rgb}{0.0, 0.0, 0.5}
\newcommand{\solution}[1]
  {\ifthenelse{\equal{\issolution}{true}}
  {\begin{quote}
    \addtocounter{taskcounter}{-1}
    \fbox{\textcolor{solutioncolor}{\bf Solution \arabic{section}.\arabic{taskcounter}}}
    \addtocounter{taskcounter}{1}
    \textcolor{solutioncolor}{\input{./solution/#1}}
  \end{quote}}
  {}}

\newcommand{\qbox}{\fbox{???}}

\newcommand{\ms}[1]{\ensuremath{\mathsf{#1}}}
\newcommand{\irl}[1]{\mathtt{#1}}


\newcommand{\lam}{\irl{lam}}
\newcommand{\fix}{\irl{fix}}
\newcommand{\parr}[2]{#1 \to #2}
\newcommand{\pair}[2]{\langle #1, #2 \rangle}

\newcommand{\natt}{\irl{nat}}
\newcommand{\prodt}[2]{#1\times#2}
\newcommand{\voidt}{\irl{void}}
\newcommand{\sumt}[2]{#1 + #2}

\newcommand{\intro}{\text{-I}}
\newcommand{\elim}{\text{-E}}

\newcommand{\val}[1]{#1~\textsf{val}}
\newcommand{\abst}[2]{#1.#2}
\newcommand{\steps}[2]{#1 \mapsto #2}
\newcommand{\subst}[3]{[#1/#2]#3}
\newcommand{\err}[1]{#1~\textsf{err}}

\newcommand{\goesto}{\hookrightarrow}
\newcommand{\hra}{\hookrightarrow}

\newcommand{\proves}{\vdash}
\newcommand{\hasType}[2]{#1 : #2}
\newcommand{\typeJ}[3]{#1 \proves \hasType{#2}{#3}}
\newcommand{\ctx}{\Gamma}
\newcommand{\emptyCtx}{\emptyset}
\newcommand{\xCtx}[2]{\ctx, \hasType{#1}{#2}}
\newcommand{\typeJC}[2]{\typeJ{\ctx}{#1}{#2}}

\newcommand{\valrule}{\textsf{v}}
\newcommand{\progrule}{\textsf{s}}
\newcommand{\elimrule}{\textsf{e}}

\newcommand{\zero}{\irl{z}}
\newcommand{\suc}[1]{\irl{s}(#1)}
\newcommand{\rec}[5]{\irl{rec}(#1) \{\irl{z} \goesto #2 \por \suc{#3} ~\irl{with}~#4 \goesto #5\} }
\newcommand{\iter}[4]{\irl{iter}(#1) \{\irl{z} \goesto #2 \por #3 \goesto #4\} }


\newcommand{\bool}{\irl{bool}}
\newcommand{\true}{\irl{true}}
\newcommand{\false}{\irl{false}}
\newcommand{\ifb}[3]{\irl{if}(#1;#2;#3)}

\newcommand{\pred}{\irl{pred}}
\newcommand{\sub}{\irl{sub}}
\newcommand{\leqf}{\irl{leq}}

\newcommand{\option}[1]{#1~\irl{opt}}
\newcommand{\none}{\irl{nothing}}
\newcommand{\some}[1]{\irl{just}(#1)}

\newcommand{\extend}{\irl{extend}}

\newcommand{\fn}[3]{\lambda(#1:#2)~#3}
\newcommand{\ap}[2]{#1(#2)}

\newcommand{\lprodt}[1]{\langle #1\rangle}
\newcommand{\lsumt}[1]{[#1]}
\newcommand{\lpair}{\pair{l_1\hra e_1}{\dots,l_n\hra e_n}}
\newcommand{\lpairprime}{\pair{l_1\hra e_1'}{\dots,l_n\hra e_n'}}
\newcommand{\lprodtn}{\lprodt{l_1\hra\tau_1,\dots,l_n\hra\tau_n}}
\newcommand{\lsumtn}{\lsumt{l_1\hra\tau_1,\dots,l_n\hra\tau_n}}
\newcommand{\lprodn}{\lprodt{l_1\hra e_1,\dots,l_n\hra e_n}}
\newcommand{\lsumn}{\lsumt{l_1\hra e_1,\dots,l_n\hra e_n}}


\newcommand{\proj}[2]{#1\cdot#2}
%\newcommand{\inj}[3]{\irl{in}[#1]\{#2\}(#3)}
\newcommand{\inj}[3]{{#2}\cdot{#3}}

\newcommand{\case}[2]{\irl{case}~#1~\irl{ of }~\{#2\}}
\newcommand{\casen}{\case{e}{\inj{}{l_1}{x_1} \hra e_1\por \dots \por \inj{}{l_n}{x_n} \hra e_n}}

\newcommand{\por}{\hspace*{.2em}|\hspace*{.2em}}
\newcommand{\letbind}[3]{\irl{let}~#1~\irl{be}~#2~\irl{in}~#3}
\newcommand{\match}[2]{\irl{match}~#1~\{#2\}}
\newcommand{\npat}{l_1~p_1\goesto e_1 \por \dots \por l_n~p_n\goesto e_n}
\newcommand{\nmatch}[1]{\match{#1}{\npat}}
\newcommand{\alias}[2]{#1~@~#2}
\newcommand{\transJ}[4]{#1\vdash#2:#3\leadsto#4}
\newcommand{\transJC}[3]{\transJ{\Gamma}{#1}{#2}{#3}}
\newcommand{\matchJC}[3]{#1:#2\goesto{#3}}
\newcommand{\matchSet}[2]{#1\implies#2}

\newcommand{\defn}{\triangleq}

\newcommand{\pimp}{\supset}

\newcommand{\nmod}{~\mathtt{mod}~}
\newcommand{\ndiv}{~\mathtt{div}~}

\newtheorem{lemma}{Lemma}
\newtheorem{theorem}{Theorem}

\newcommand{\problemset}[1]
  {\ifthenelse{\equal{\issolution}{true}}
  {}{{#1}}}

\setcounter{taskcounter}{1}
\setcounter{taskPercentCounter}{0}
\setcounter{taskcounterSection}{\value{section}}

%\newcommand{\ttt}[1]{\texttt{#1}}

%% part of a problem
\newcommand{\task}[1]
  {\bigskip \noindent {\bf Task\mayresettaskcounter{}\addtocounter{taskPercentCounter}{#1} \arabic{section}.\arabic{taskcounter}\addtocounter{taskcounter}{1}}.}

\newcommand{\ectask}
  {\bigskip \noindent {\bf Task\mayresettaskcounter{} \arabic{section}.\arabic{taskcounter}\addtocounter{taskcounter}{1}} (Extra Credit).}

%% The rule counter
\newcounter{rule}
\setcounter{rule}{0}
\newcommand{\rn}
  {\addtocounter{rule}{1}(\arabic{rule})}


\title{OPLSS 2017 PL Background Day 1}

\author{Robert Harper and Dan Licata}
\date{}

\begin{document}
\maketitle

\begin{quote} 
\emph{These exercises are based on materials for Carnegie Mellon 15-312:
  Principles of Programming Languages; we thank many intructors and
  teaching assistants for contributing to the design and implementation
  of these problems.}
\end{quote}

Here are some problems you can use to review the first day's materials.
Please note that we definitely do \emph{not} expect you to finish all of
these before tomorrow!  Students come to OPLSS with very different
backgrounds, and our only goal is that \emph{everyone} learns
\emph{something}.  If you haven't seen much of this material before, you
might spend several days's hands-on sessions working on this assignment.
If you've seen the basics before, there are more advanced problems later
in this assignment, and more research-level courses coming up later in
the week.

The online preview of PFPL is here:
\begin{center}
\url{http://www.cs.cmu.edu/~rwh/pfpl/2nded.pdf}
\end{center}


\paragraph{Paper and pencil}

\begin{enumerate}

\item Problem 1: This problem asks you to write some programs in
  G\"odel's T.

\item Problem 2: This problem asks you to prove determinism and the
  correspondence between an evaluation relation and a structural
  operational semantics.  If you have never done proofs by rule
  induction before, start here.

\item Problem 3: This problem asks you to do some type safety (progress
  and preservation) proofs and to learn about product and sum types.  If
  you haven't done many proofs by rule induction, or haven't seen
  progress and preservation before, you should start here, before doing
  the logical relations proofs.  

\item Problem 4: This problem asks you to practice some of the materials
  from the logical relations lecture.
\end{enumerate}

\paragraph{Implementation}
Problem 5: This problem asks you to implement a type checker and
operational semantics for a simple functional programming language with
labeled sums and products (see PFPL Chapters 10 and 11 specifically).
If you haven't seen type systems and operational semantics before,
implementing them is a good way to really understand them.  If you've
written a type checker or an interpreter for a functional language, your
time might be better spent on the logical relations proofs.

\section{Programming in G\"odel's T}

For this section, work in G\"odel's T (see PFPL Chapter 9).  

\task{0} Define an addition function $\mathtt{plus} : \natt \to \natt \to \natt$.

\task{0} Define a multiplication function $\mathtt{mult} : \natt \to \natt \to \natt$.

\task{0} Define a factorial function $\mathtt{fact} : \natt \to \natt$.

\task{0} It is possible to define G\"odel's T with either
\emph{iteration} or \emph{recursion}.  The difference is

\[
\inferrule{
\typeJC{e}{\natt}\\
\typeJC{e_z}{\tau}\\
\typeJ{\ctx, \hasType{x}{\natt},\hasType{y}{\tau}}{e_1}{\tau}
}{
\typeJC{\rec{e}{e_z}{x}{y}{e_1}}{\tau}
}
\qquad
\inferrule{
\typeJC{e}{\natt}\\
\typeJC{e_z}{\tau}\\
\typeJ{\ctx,\hasType{y}{\tau}}{e_1}{\tau}
}{
\typeJC{\iter{e}{e_z}{y}{e_1}}{\tau}
}
\]

That is, in ``iteration'', the successor case does not bind a variable
standing for the predecessor of the number, only for the recursive call,
while in ``recursion,'' it does bind a variable standing for the
predecessor.  Iteration is clearly a special case of recursion,  where
the variable goes unused.

Perhaps surprisingly, it is also possible to show that conversely,
recursion can be defined from iteration.  Show how to code recursion
using iteration in G\"odel's T \emph{with product types} (do the next
problem first if you don't know how those work): define the recursor and
show that the instruction steps for executing the recursor on $\zero$
and $\suc{n}$ are simulated by one or more instruction steps in the
language with iteration.

\section{Rule Induction Warm-Up}

For this problem, again consider G\"odel's T as described in PFPL
Chapter 9, i.e. a language with function types $\to$ and natural
numbers.

\paragraph{Determinism of Stepping}

\task{0} Prove that stepping is deterministic:
\begin{quote}
For all $e$, if $e \mapsto v$ and $e \mapsto v'$ then $v =_\alpha v'$
($v$ and $v'$ are equal values, where equal means $\alpha$-equivalent).
\end{quote}

\paragraph{Determinism of Multi-step Evaluation}

Define multi-step evaluation as the \emph{reflexive-transitive closure}
of evaluation:
\[
\inferrule{ }
          {e \mapsto e}
\qquad
\inferrule{e \mapsto e' \\
           e' \mapsto^* e''}
          {e \mapsto^* e''}
\]

\task{0} Prove that multi-stepping is deterministic:
\begin{quote}
For all $e$, if $e \mapsto^* v$ and $e \mapsto^* v'$ then $v =_\alpha
v'$ ($v$ and $v'$ are equal values, where equal means $\alpha$-equivalent).
\end{quote}

\paragraph{Relating SOS and Evaluation}

One way to define ``evaluating a program all the way to a value'' is
defining structural operational semantics $e \mapsto e'$, and then
defining multistep evaluation as $e \mapsto^* e'$.  Another is to
directly define an \emph{evaluation relation} $e \Downarrow v$.  For
example, for call-by-name we have the following rules for $\to$:
\[
\inferrule{  }
          { (\lambda x:\tau.e) \Downarrow (\lambda x:\tau.e) }
\qquad
\inferrule{ e_1 \Downarrow (\lambda x:\tau.e) \\
            e[e_2/x] \Downarrow v \\
          }
          { (e_1 \, e_2) \Downarrow v }
\]

\task{0} Prove that the evaluation relation and multi-step SOS correspond:
\begin{quote}
$e \Downarrow v$ iff $e \mapsto^* v$
\end{quote}

You will need some lemmas; if you get stuck, see PFPL Section 7.2 for a
hint.

\section{Progress and Preservation}

If you haven't seen progress and preservation proofs before, first read
the proof for functions and application in PFPL section 8.2.    

\paragraph{Natural Numbers}

\task{0} Prove progress and preservation for the natural numbers in
G\"odel's T, i.e. do the cases of progress and preservation for $0$,
successor, and the recursor (recursion).

\paragraph{Products}

Read PFPL Section 10.1, and then and prove Theorem 10.1, progress and
preservation for a language with pair types $\tau_1 \times \tau_2$.

\task{0}
Prove preservation, that is:
 \begin{quote}
 If $\typeJ{\cdot}{e}{\tau}$ and $\steps{e}{e'}$ then $\typeJ{\cdot}{e'}{\tau}$.
 \end{quote}

 \task{0} Prove progress. That is:
 \begin{quote}
 If $\typeJ{\cdot}{e}{\tau}$ then either $\val{e}$ or there is some $e'$ such that $\steps{e}{e'}$.
 \end{quote}

\paragraph{Sums}

\task{0} Read PFPL Section 11.1 and prove Theorem 11.1, progress and
preservation for the sum type $\tau_1 + \tau_2$.  

\section{Logical Relations}

\newcommand\logrel[2]{\ensuremath{\mathsf{Good}_{#1}(#2)}}

In lecture, we defined a logical relation and used it to prove
termination.  The relation \logrel{\tau}{e} is a predicate on closed
terms $\typeJ{\cdot}{e}{\tau}$, and is defined by induction on $\tau$ as

\begin{itemize}
\item $\logrel{\irl{int}}{e}$ iff there exists a number $k$ such that $e
  \mapsto^* \mathsf{num}[k]$

\item $\logrel{\tau_2 \to \tau}{f}$ iff 
  there exists a term $x : \tau_2 \vdash e : \tau$ 
  such that (1) $f \mapsto^* (\lambda x.e)$
  and (2) for all $\typeJ{\cdot}{e_2}{\tau_2}$, $\logrel{\tau_2}{e_2}$ implies
  $\logrel{\tau}{e[e_2/x]}$.
\end{itemize}

A substitution $\typeJ{\cdot}{\theta}{\Gamma}$ is a list
$(e_1/x_1,\ldots e_n/x_n)$ where $\typeJ{\cdot}{e_i}{\tau_i}$ for each
variable $x_i:\tau_i$ in $\Gamma$.  We write $e[\theta]$ for performing
the substitution $\theta$, plugging in for all of the variables in $e$.
For a substitution $\typeJ{\cdot}{\theta}{\Gamma}$, we say
$\logrel{\Gamma}{\theta}$ iff $\logrel{\tau}{\theta(x)}$ for all
$x:\tau$ in $\Gamma$.

The fundamental theorem says that well-typed terms preserve goodness:
\begin{quote}
If $\typeJ{\Gamma}{e}{\tau}$ and $\logrel{\Gamma}{\theta}$ then
$\logrel{\tau}{e[\theta]}$.  
\end{quote}

\task{0} Extend the languages with binary pairs as in PFPL Section 10.1
with a lazy operational semantics.  This means that $\langle e_1, e_2
\rangle$ is always a value, and does not evaluate $e_1$ and $e_2$ until
a projection is taken.  Check that closure under converse evaluation
still holds, and do the cases of the fundamental theorem for pairing and
projections.

\task{0} Extend the definition of the logical relation to sum types
$\tau_1 + \tau_2$, again with a lazy operational semantics (so both
$\mathsf{inl}(e)$ and $\mathsf{inr}(e)$ are values regardless of whether
$e$ is).  Check that closure under converse evaluation still holds, and
do the cases of the fundamental theorem for injections and case
analysis.

\task{0} Replace the base type $\irl{int}$ with natural numbers as in
System T (again with a lazy successor, so $\suc{e}$ is always a value).
Define the logical relation for $\natt$ by an inner or ``horizontal''
induction:

\[
\infer{\logrel{\natt}{e}}
      {e \mapsto^* \zero}
\qquad
\infer{\logrel{\natt}{e}}
      {e \mapsto^* \suc{e_0} & 
        \logrel{\natt}{e_0}
      }
\]

You will need to do an inner induction on ${\logrel{\natt}{e}}$ to
prove the case of the fundamental theorem for the recursor.    

\task{0} Adapt all of the above to call-by-value/eager instead of
call-by-name/lazy. What changes in the proofs?

%% Note that some of these exercises are solved in PFPL Chapter 46, so look
%% there if you get stuck.

\section{Implementing System T with Labeled Sums and Products}

For this section of the assignment, we will be working on System T
extended with labeled sum and product types.  The abstract syntax is as
follows; on the left is the formal abstract binding tree notation, and
on the right is a more reader-friendly display notation, which should be
thought of as synonyms for the left-hand column.  

\[
\begin{array}{r c c l l l}
\ms{Type} & \tau & ::=
			&\irl{nat}                	 				& \irl{nat} 													& \text{naturals}\\
         &	&	&\irl{arr}(\tau_1;\tau_2) 				& \parr{\tau_1}{\tau_2} 										& \text{function}\\
         & 	&	&\irl{prod}(l_1\hra\tau_1;\dots;l_n\hra\tau_n)		& \lprodtn											& \text{product}\\
      	 & 	&	&\irl{sum}(l_1\hra\tau_1;\dots;l_n\hra\tau_n)		& \lsumtn											& \text{sum}\\
	 \\
\ms{Exp}
        & e   & ::= 	& x                                				& x 														& \text{variable}\\
        &	&	& \zero							& \zero													& \text{zero}\\
        &      &      	& \suc{e}                     				& \suc{e}													& \text{successor} \\
        &      &      	& \irl{rec}\{e_z;x.y.e_1\}(e)   			& \rec{e}{e_z}{x}{y}{e_1}										& \text{recursion}\\
        &      &      	& \lam\{\tau\}(x.e)           				& \fn{x}{\tau}{e}												& \text{abstraction} \\
        &      &     	& \irl{ap}(e_1;e_2) 					& \ap{e_1}{e_2} 											& \text{application}\\
        &      &	& \irl{tuple}(l_1\hra e_1;\dots;l_n\hra e_n) 	& \lpair													& \text{tuple}\\
 	&	&	& \irl{proj}[l](e)						& \proj{e}{l}												& \text{projection}\\
	&	&	& \irl{in}[l]\{\tau\}(e)					& \inj{\tau}{l}{e}												& \text{injection}\\
	&	&	& \irl{case}[l_1,\dots,l_n](x_1.e_1;\dots;x_n.e_n)(e) 	& \casen											& \text{casing}\\
\end{array}
\]
Labels, represented by $l$ in the grammar, are simply identifiers
associated with elements of tuples and sums to give them some
description and allow for a more intuitive elimination form.

The types of this language are natural numbers and function types,
together with labeled products and and labeled sums.  

A labeled product is a \emph{record} with named fields.  To create a
record, you give a value for each field.  To use a record, you project a
specified field.  Records are included in some form or another in most
functional languages (e.g. ML, Haskell).  Records are somewhat analogous
to the fields of an object in object-oriented programming languages;
additionally, many languages have a notion of a function with named
arguments, which can be modeled by a function that takes a record as an
argument.

More formally, the type $\lprodtn$ is a labeled product type whose
component labeled $l_i$ has type $\tau_i$.  The introduction form for
labeled products has the form $\lprodn$ whose $l_i$ component is $e_i$.
The projection $\proj{e}{l}$ selects the $l$ component of the labeled
tuple $e$.
\begin{mathpar}
\small
\inferrule{
	\typeJC{e_1}{\tau_1}~\dots~\typeJC{e_n}{\tau_n}\\
	\forall i\neq j.l_i\neq l_j
}{
	\typeJC{\lpair}{\lprodtn}
}(\prodt{}\intro)

\inferrule{
	\typeJC{e}{\lprodtn}\\
	(1\leq i\leq n)
}{
	\typeJC{\proj{e}{l_i}}{\tau_i}
}(\prodt{}\elim)

\inferrule{
	\val{e_1}~\dots~\val{e_n}
}{
	\val{\lprodn}
}(\irl{tuple}_\valrule)
\quad
\inferrule{
	\val{e_1}\dots \val{e_{i-1}}\\
	\steps{e_i}{e_i'}\\
}{
	\steps{\lprodn}{\lprodt{l_1\hra e_1,\dots,l_i\hra e_i',\dots,l_n\hra e_n} }
}(\irl{tuple}_\progrule)

\inferrule{
	\steps{e}{e'}
}{
	\steps{e.l_i\{\tau\}}{e'.l_i\{\tau\}}
}(\irl{proj}_\progrule^1)
\qquad
\inferrule{
	\val \lpair
}{
	\steps{\proj{\lprodn}{l_i}}{e_i}
}(\irl{proj}_\progrule^2)

\end{mathpar}

Dually, a labeled sum type is a like a non-recursive datatype in ML or
Haskell.  To make a value of this type, you choose a label and supply
the appropriate payload data.  To use a value of this type, you
case-analyze the possible ways it could have been constructed.  The type
$\lsumtn$ is the labeled sum type whose elements are elements of type
$\tau_i$ labeled with $l_i$, for some $1\leq i\leq n$. The injection
$\irl{inj}\{\tau\}(l;e)$ creates a sum of type $\tau$ with $e$ at label
$l$. $\casen$ takes a sum and if it is of form $l_i$, binds the contents
to $x_i$ in $e_i$.
\begin{mathpar}
\small
\inferrule{
	\typeJC{e}{\tau_i}\\
	\forall i\neq j.l_i\neq l_j
}{
	\typeJC{\inj{}{l_i}{e}}{\lsumtn}
} (\sumt{} \intro)

\inferrule{
	\typeJC{e}{\lsumtn}\\
	\typeJ{\Gamma,\hasType{x_i}{\tau_i}}{e_i}{\tau}
	~\dots~
	\typeJ{\Gamma,\hasType{x_i}{\tau_i}}{e_i}{\tau}
}{
	\typeJC{\casen}{\tau}
}(\sumt{} \elim)

\inferrule{
	\val{e}
}{
	\val{\inj{\tau}{l}{e}}
}(\irl{inj}_\valrule)
\quad
\inferrule{
	\steps{e}{e'}
}{
	\steps{\inj{\tau}{l}{e}}{\inj{\tau}{l}{e'}}
}(\irl{inj}_\progrule)

\inferrule{
	\steps{e}{e'}
}{
	\steps{\case{e}{l_1(x_1) \goesto e_1 \por \dots \por l_n(x_n)\goesto e_n}}{\case{e'}{l_1(x_1) \goesto e_1 \por \dots \por l_n(x_n)\goesto e_n}}
}(\irl{case}_\progrule^1)

\inferrule{
	\val{\inj{}{l_i}{e}}
}{
	\steps{\case{\inj{\tau}{l_i}{e}}{l_1(x_1) \goesto e_1 \por \dots \por l_n(x_n)\goesto e_n}}
		 {[e/x_i]e_i}
}(\irl{case}_\progrule^2)

\end{mathpar}

Your goal in this section will be to implement the static and dynamic
semantics of System T with labeled sums and products.  The full rules
are included in Section~\ref{sec:full} for your reference.

You are welcome to do this in any language if you want to start from
scratch.  However, if you use Standard ML, we have provided a lot of
definitions and scaffolding that you will find useful.  

\paragraph{Data Structures} The directory \texttt{cmlib} contains a general SML library.
You may want to use some operations from the signatures in
\texttt{dict.sig} and \texttt{set.sig} for dictionaries and sets.

\paragraph{Names}  One common issue in implementing programming
languages is representing names/variables/identifiers in such a way that
you can check $\alpha$-equivalence of syntax trees and correctly
implement (capture-avoiding) substitution.

In the \texttt{var} subdirectory, we have provided an abstract type of
of "temp"oraries/names, which are used to implement variables and
labels.  The file \texttt{var/temp.sig} describes the operations on
names.  Key operations include:

\begin{code}
signature TEMP =
sig
  type t

  (* Creates a new, globally unique temp. *)
  val new : string -> t

  (* Tests whether two temps are equal. *)
  val equal  : (t * t) -> bool

  ...
end
\end{code}

The type \texttt{t} is an abstract/private/unknown type with the
operations provided in the rest of the signature.  The operation
\texttt{new} creates a new \emph{globally unique} name (i.e. every time
you call \texttt{new}, you are guaranteed that the name provided will be
different than any other name that already exists).  Though it is not
necessary (e.g. you could just as well always call \verb|new ""|,
because each new name is different), for readability, \texttt{new} takes
a string that is used when the name is printed out.  Names can be
compared for equality using the \texttt{equal} function.

In your implementation, the modules Label and Variable have signature
\texttt{TEMP}.

\paragraph{Abstract binding trees}  Next, we have implemented the
syntax of types and terms; see \sml{labT/labt.sig} for the interface.  

As a first cut, you might expect that we would represent terms with a datatype
like
\begin{code}
    datatype term
      = Var of termVar
      | Lam of (termVar * Typ.t) * term
      | Ap of term * term
\end{code}
where \texttt{termVar} is a type of names as described above.  This says
that a \texttt{Lam} term has a variable, a type ascription for the
variable, and another term standing for the body of the function; an
application has two subterms, for the function and the argument.  A
problem with this representation is that, as you traverse a term, you
need to remember to substitute fresh variables at appropriate times to
avoid accidental problems with $\alpha$-equivalence and substitution,
because the same \texttt{termVar} might be used in many places with
different meanings.

Instead, we will use an abstract type to hide all of this variable
manipulation in the implementation of terms.  Here is the signature:

\begin{verbatim}
  structure Term : sig
    type termVar = Variable.t
    type term

    datatype view
      = Var of termVar
      | Z
      | S of term
      | Lam of (termVar * Typ.t) * term
      | Ap of term * term
      | Pair of (Label.t * term) list
      | Proj of term * Label.t
      | Inj of Typ.t * Label.t * term
      | Case of term * ((Label.t * termVar) * term) list
      | Iter of term * term * (termVar * (termVar * term))

    val into : view -> term
    val out : term -> view

    val aequiv : term * term -> bool
    val subst : term -> termVar -> term -> term

    ... 

  end
\end{verbatim}

The type \sml{term} is abstract, while the type \sml{view} is a datatype
like above---but not that the subtrees are abstract \sml{term}s, not
recursive occurences of \sml{view}.  There are functions \sml{into} and
\sml{out} that map between these, so to \emph{create} a term, you can
call \sml{into} on a \sml{view}, while to case-analyze a term (as you
will do in your type checker and operational semantics), you call
\sml{out}.  The key idea is:
\begin{quote}
\emph{whenever \sml{out} exposes a \sml{termVar} in the view (e.g. in a
  \sml{Lam} or \sml{Case} or \sml{Iter}), that variable is \emph{fresh}
  (distinct from all other variables in play)}
\end{quote}
Therefore, in your code, you can assume all variables are
fresh/distinct, which simplifies the implementation.  

For conveneince, the \sml{Term} module also provides functions named
\sml{Var'} and \sml{Z'} and \sml{S'} and \sml{Lam'} \ldots, where
\sml{C'} abbreviates \sml{into(C ...)} for the corresponding view
constructor.

The \sml{Term} module also provides implementations of
$\alpha$-equivalence and substitution for your convenience.  

\paragraph{Your code}

Your code will go in the \sml{sem} directory, in \sml{typechecker.sml}
and \sml{dynamics.sml}, which implement the signatures in the
corresponding \sml{.sig} files.  

\task{0} Implement the statics of the language. You will need to
implement the module \verb|TypeChecker|, which can be found in
``\verb|sem/typechecker.sml|''.

Hints:
\begin{itemize}
\item Use the structure \sml{Context} for dictionaries mapping term
  variables to types.

\item Your main goal is to implement the function \sml{checkType},
which takes a context and a term and \emph{synthesizes} a type if the
expression is well-typed.  If the expression is ill-typed, the function
should raise the exception \sml{TypeError}.  

\item The helper function \sml{equiv e t1 t2} checks if the types
  \sml{t1} and \sml{t2} are equal. If so, it returns \sml{()}, the one
  value of the \sml{unit} type, and if not, it raises a \sml{TypeError}
  signalling that there was a problem type checking \sml{e} (which is
  provided only for error reporting).  A common SML idiom is 
\begin{verbatim}
let val () = equiv e t1 t2 
 in rest
 end 
\end{verbatim}
which signals that \sml{equiv e t1 t2} is being run only for its
computational effect of possibly raising an exception, and when it
succeeds, we run \sml{rest}.  This can also be written
\begin{verbatim}
equiv e t1 t2; rest
\end{verbatim}

\item Your type checker should ensure that all types occuring in type
  checking are well-formed, which in this case means only that the
  labeled sums and products have no duplicate labels.  

\end{itemize}

\task{0} Implement the dynamics of the language. You will need to
implement the module \verb|LabTDynamics|, which can be found in
``\verb|sem/dynamics.sml|''. 

 Your overall goal is to implement the function
\begin{code}
  datatype d = STEP of Term.term | VAL
  val trystep : Term.term -> d
\end{code}
(the rest of the module is implemented for you).  This should satisfy
the following specification:
\begin{quote}
If \sml{e} is a closed, well-typed term, then \sml{trystep e} returns
\sml{STEP e'} if $\sml{e} \mapsto \sml{e'}$, or \sml{VAL} if \sml{e} is
a value.
\end{quote}

Hints:
\begin{itemize}
\item \sml{trystep} is an \emph{implementation of the progress proof for
  the language}.  If you get stuck on how to proceed, try doing the
  corresponding case of the proof.

\item Ignore the exceptions \sml{RuntimeError} and \sml{Abort} and the
  type \sml{D}, which are not used here (they are for other uses of this
  support code).

\item Because we are \emph{assuming} and \sml{e} is well-typed (i.e. we
  run the type checker before running the operational semantics), your
  implementation can do anything it wants if it encounters an ill-typed
  term---there is no need to specifically test for such cases and report
  them. However, we have provided an exception \sml{Malformed} that
  could be used in cases where you do discover that the term is
  ill-typed.
\end{itemize}

\paragraph{Testing}

To compile your program, start \sml{sml} from the handout directory and
then do
\begin{verbatim}
- CM.make "sources.cm";
\end{verbatim}

Our support code provides a \emph{read-typecheck-eval-print-loop} (REPL)
that you can use to test interactively.  To start it, run
\begin{verbatim}
- TopLevel.repl()
\end{verbatim}

The REPL keeps tract of a current expression that you are working.
There are three commands that you can use
\begin{itemize}
\item \sml{load e;}.  Type checks the expression \sml{e} and sets it as
  the current expression to be evaluated.  
\item \sml{step;} steps the current expression one step, and reports the
  new expression or if the current expression is a value.  
\item \sml{eval;} evaluates the current expression all the way to a
  value.  
\end{itemize}
The commands \sml{step} and \sml{eval} can also be called with an
expression argument, which is like running \sml{load e} before them.

Here are some examples that show the concrete syntax (more examples are
in the \sml{tests} directory):

\begin{enumerate}
\item 
\begin{verbatim}
->load s(z);
Statics: term has type Nat
 |--> (S Z)
->step;
 (S Z) VAL
->step;
Error: Nothing to step!
\end{verbatim}

\item 
\begin{verbatim}
->eval fn (x:nat) x;
Statics: term has type (Arrow (Nat, Nat))
Statics: term has type (Arrow (Nat, Nat))
 (Lam ((x@13, Nat) . x@13)) VAL
\end{verbatim}

\item 
\begin{verbatim}
->load (fn (x:nat) x) z;
Statics: term has type Nat
 |--> (Ap ((Lam ((x@18, Nat) . x@18)), Z))
->step;
Statics: term has type Nat
 |--> Z
->step;
 Z VAL
\end{verbatim}

\item Type errors should be reported:   
\begin{verbatim}
->load s(fn (x:nat) x);
LabTChecker error in term: (S (Lam ((x@38, Nat) . x@38)))
\end{verbatim}

\item $\rec{e}{e_0}{x}{p}{e_1}$ is written \sml{iter e \{ e0 | s(x) with p => e1\}}.  
\begin{verbatim}
->eval (fn (x:nat) iter x { z | s(y) with p => s(s(p))}) (s(s(z)));
Statics: term has type Nat
Statics: term has type Nat
 (S (S (S (S Z)))) VAL
\end{verbatim}

\item Record values are written \sml{< l1 = e1 , ... , ln = en>}.  

\begin{verbatim}
->eval <bar = z, foo = s(z)>.foo;
Statics: term has type Nat
Statics: term has type Nat
 (S Z) VAL
\end{verbatim}

\item Because types are inferred, sum injections must be annotated with
  all of the constructors (a better type checker would infer these).
  For example, what we would write as $\inj{}{\mathsf{SOME}}{e}$ in the
  labeled sum type 
  \[
  \lsumt{\mathsf{NONE} \hra \langle\rangle, \mathsf{SOME} \hra \natt}
  \]
  is written 
\begin{verbatim}
   in[NONE :: <>, SOME :: nat]{SOME}(z)
\end{verbatim}

\item 
\begin{verbatim}
->eval case (in[NONE :: <>, SOME :: nat]{SOME}(z))
=>          {NONE m => z | SOME w => s(z)};
Statics: term has type Nat
Statics: term has type Nat
 (S Z) VAL
\end{verbatim}
\end{enumerate}

You can also run an example from a file using \sml{TopLevel.evalFile
  filename}, and there is an automated testing harness in
\verb|tests.sml|.  


\section{Appendix: Full Rules for G\"odel's T with labeled products and sums}

\label{sec:full}

\subsection{Statics}

\subsubsection{System T}\label{tstatics}
\begin{mathpar}
\small
\inferrule{
 }{\typeJ{\Gamma,x:\tau}{x}{\tau}} (\text{var})

\inferrule{ }{\typeJC{\zero}{\natt}} (\natt\intro_1)
\qquad
\inferrule{
\typeJC{e}{\natt}
}{
\typeJC{\suc{e}}{\natt}
} (\natt\intro_2)
\qquad
\inferrule{
\typeJC{e}{\natt}\\
\typeJC{e_z}{\tau}\\
\typeJ{\ctx, \hasType{x}{\natt},\hasType{y}{\tau}}{e_1}{\tau}
}{
\typeJC{\rec{e}{e_z}{x}{y}{e_1}}{\tau}
} (\irl{rec})

\inferrule{
\typeJ{\ctx, \hasType{x}{\tau_1}}{e}{\tau_2}
}{
\typeJC{\lam\{\tau_1\}(x.e)}{\parr{\tau_1}{\tau_2}}
} (\parr{}{}\intro)
\qquad
\inferrule{
\typeJC{e_1}{\parr{\tau_1}{\tau_2}}\\
\typeJC{e_2}{\tau_1}
}{
\typeJC{\ap{e_1}{e_2}}{\tau_2}
} (\parr{}{}\elim)

\end{mathpar}

\subsubsection{Labeled Sums and Products}\label{prodstatics}
\begin{mathpar}
\small
\inferrule{
	\typeJC{e_1}{\tau_1}~\dots~\typeJC{e_n}{\tau_n}\\
	\forall i\neq j.l_i\neq l_j
}{
	\typeJC{\lpair}{\lprodtn}
}(\prodt{}\intro)
\qquad
\inferrule{
	\typeJC{e}{\lprodtn}\\
	(1\leq i\leq n)
}{
	\typeJC{\proj{e}{l_i}}{\tau_i}
}(\prodt{}\elim)

\inferrule{
	\typeJC{e}{\tau_i}\\
	\forall i\neq j.l_i\neq l_j
}{
	\typeJC{\inj{}{l_i}{e}}{\lsumtn}
} (\sumt{} \intro)

\inferrule{
	\typeJC{e}{\lsumtn}\\
	\typeJ{\Gamma,\hasType{x_i}{\tau_i}}{e_i}{\tau}
	~\dots~
	\typeJ{\Gamma,\hasType{x_i}{\tau_i}}{e_i}{\tau}
}{
	\typeJC{\casen}{\tau}
}(\sumt{} \elim)

\end{mathpar}

\subsection{Dynamics (Eager, Left-to-Right)}\label{dynamics}

\subsubsection{System T}\label{tdynamics}
\begin{mathpar}
\small
\inferrule{ }{\val{\zero}} (\natt_\valrule^1)

\inferrule{
  \val{e}
}{
  \val{\suc{e}}
} (\natt_\valrule^2)

\inferrule{ }{\val{\lam[\tau](x.e)}} (\parr{}{}_\valrule)

\inferrule{
   \steps{e}{e'}
}{
   \steps{\suc{e}}{\suc{e'}}
} (\natt_\progrule)

\inferrule{
   \steps{e_1}{e_1'}
}{
   \steps{e_1~e_2}{e_1'~e_2}
} (\text{ap}_\progrule^1)

\inferrule{
  \val{e_1}\\
  \steps{e_2}{e_2'}
}{
   \steps{e_1~e_2}{e_1~e_2'}
} (\text{ap}_\progrule^2)

\inferrule{\val{e_2}}
{
\steps{\lam[\tau](x.e)~e_2}{[e_2/x]e}
} (\text{ap}_\elimrule)

\inferrule{\steps{e}{e'}}{\rec{e_z}{x}{y}{e_1}{e}}(\irl{rec}_\progrule^1)

\inferrule{ }{\steps{\rec{e_z}{x}{y}{e_1}{z}}{e_z}}(\zero)(\irl{rec}_\progrule^2)

\inferrule{ \val{e}}{\steps{\rec{e_z}{x}{y}{e_1}{\suc e}}{[e,\rec{e_z}{x}{y}{e_1}{e}/x,y]e_1}}(\irl{rec}_\progrule^3)
\end{mathpar}

\subsubsection{Labeled Products and Sums}\label{proddynamics}
\begin{mathpar}
\small

\inferrule{
	\val{e}
}{
	\val{\inj{\tau}{l}{e}}
}(\irl{inj}_\valrule)

\inferrule{
	\steps{e}{e'}
}{
	\steps{\inj{\tau}{l}{e}}{\inj{\tau}{l}{e'}}
}(\irl{inj}_\progrule)

\inferrule{
	\steps{e}{e'}
}{
	\steps{\case{e}{l_1(x_1) \goesto e_1 \por \dots \por l_n(x_n)\goesto e_n}}{\case{e'}{l_1(x_1) \goesto e_1 \por \dots \por l_n(x_n)\goesto e_n}}
}(\irl{case}_\progrule^1)
\\

\inferrule{
	\val{\inj{}{l_i}{e}}
}{
	\steps{\case{\inj{\tau}{l_i}{e}}{l_1(x_1) \goesto e_1 \por \dots \por l_n(x_n)\goesto e_n}}
		 {[e/x_i]e_i}
}(\irl{case}_\progrule^2)

\\

\inferrule{
	\val{e_1}~\dots~\val{e_n}
}{
	\val{\lprodn}
}(\irl{tuple}_\valrule)
\qquad
\inferrule{
	\val{e_1}\dots \val{e_{i-1}}\\
	\steps{e_i}{e_i'}\\
}{
	\steps{\lprodn}{\lprodt{l_1\hra e_1,\dots,l_i\hra e_i',\dots,l_n\hra e_n} }
}(\irl{tuple}_\progrule)

\inferrule{
	\steps{e}{e'}
}{
	\steps{e.l_i\{\tau\}}{e'.l_i\{\tau\}}
}(\irl{proj}_\progrule^1)
\qquad
\inferrule{
	\val \lpair
}{
	\steps{\proj{\lprodn}{l_i}}{e_i}
}(\irl{proj}_\progrule^2)

\end{mathpar}

\end{document}
